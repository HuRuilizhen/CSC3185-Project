%----------------------------------------------------------------------------------------
%	PACKAGES AND OTHER DOCUMENT CONFIGURATIONS
%----------------------------------------------------------------------------------------

% Pakcages
\documentclass[12pt,a4paper]{report}
\usepackage[english]{babel}
\usepackage[utf8x]{inputenc}
\usepackage[UTF8]{ctex}
\usepackage{amsmath}
\usepackage{graphicx}
\usepackage{float}
\usepackage[colorinlistoftodos]{todonotes}
\usepackage{listings}
\usepackage{xcolor}
\usepackage{tikz} \usetikzlibrary{shapes,arrows,positioning, arrows.meta}
\usepackage{fancyhdr}
\usepackage{titlesec}
\usepackage{tocloft}
\usepackage{subcaption}
\usepackage{setspace}
\usepackage[a4paper,top=3cm,bottom=2cm,left=3cm,right=3cm,marginparwidth=1.75cm]{geometry}

% Define custom colors
\definecolor{ghbggray}{HTML}{F6F8FA}
\definecolor{ghblue}{HTML}{0366D6}
\definecolor{ghgreen}{HTML}{22863A}
\definecolor{ghred}{HTML}{B31D28}
\definecolor{ghdarkgray}{HTML}{24292E}
\definecolor{ghorange}{HTML}{D73A49}

% Define code listing settings
\lstset{
    backgroundcolor=\color{ghbggray},
    commentstyle=\color{ghgreen},
    keywordstyle=\color{ghblue},
    numberstyle=\tiny\color{ghdarkgray},
    stringstyle=\color{ghred},
    basicstyle=\ttfamily\footnotesize\color{ghdarkgray},
    breakatwhitespace=false,
    breaklines=true,
    captionpos=b,
    keepspaces=true,
    numbers=left,
    numbersep=5pt,
    showspaces=false,
    showstringspaces=false,
    showtabs=false,
    tabsize=2,
    identifierstyle=\color{ghorange}
}

% Define custom page style
\renewcommand{\thesection}{\arabic{section}}
\renewcommand{\cftsecfont}{\bfseries}
\titleformat{\section}[block]{\normalfont\Large\bfseries}{\thesection}{1em}{}
\setlength{\parindent}{0pt}

% Define line spacing
\setstretch{1}

%----------------------------------------------------------------------------------------
%	COVER
%----------------------------------------------------------------------------------------

\begin{document}

\begin{titlepage}

    \newcommand{\HRule}{\rule{\linewidth}{0.5mm}}

    \center
    \vspace*{1.5cm}

    %----------------------------------------------------------------------------------------
    %	HEADING SECTIONS
    %----------------------------------------------------------------------------------------

    \includegraphics[scale=.2]{src/cuhk.png}\\[1cm]
    \textsc{\large The Chinese University of Hong Kong, Shenzhen}\\[1.5cm]

    %course code
    \textsc{\Large CSC3185}\\[0.5cm]

    %course name
    \textsc{\large Introduction to Multimedia Systems}\\[0.5cm]

    %----------------------------------------------------------------------------------------
    %	TITLE SECTION
    %----------------------------------------------------------------------------------------

    \HRule \\[0.4cm]
    { \bfseries Application of 3D Image Recognition in Medical Diagnosis}
    \HRule \\[1.5cm]

    %----------------------------------------------------------------------------------------
    %	AUTHOR SECTION
    %----------------------------------------------------------------------------------------

    \begin{minipage}{0.6\textwidth}
        \begin{tabular}{l l l}
            \emph{Group ID:}     & 4                        \\
            \emph{Group Member:} & Hong Kun     & 120040069 \\
                                 & Hu Ruilizhen & 122090168 \\
                                 & Hong Yixi    & 119010101 \\
        \end{tabular}

    \end{minipage}\\[2cm]

    %----------------------------------------------------------------------------------------
    %	DATE SECTION
    %----------------------------------------------------------------------------------------

    % Date
    {\large \today}\\[2cm]
    \vfill
\end{titlepage}

%----------------------------------------------------------------------------------------
%	CONTENTS
%----------------------------------------------------------------------------------------

% \tableofcontents
% \newpage

%----------------------------------------------------------------------------------------
%	HEADER AND FOOTER
%----------------------------------------------------------------------------------------

\fancypagestyle{mypagestyle}{
    \fancyhf{}
    \fancyhead[L]{\small CSC3185 Midterm Report}
    \fancyhead[R]{\small Group 4}
    \fancyfoot[C]{\thepage} % Add this line to include page numbers at the center of the footer
    \renewcommand{\headrulewidth}{1pt}
}

% Apply custom page style
\pagestyle{mypagestyle}

%----------------------------------------------------------------------------------------
%	MAIN BODY
%----------------------------------------------------------------------------------------

\section{Topic and Understanding}

\qquad In recent years, the application of pattern recognition technology has penetrated into various fields, opening up new possibilities for the development of artificial intelligence.
One of the most influential and dynamic application areas is in the field of health care. Especially in disease detection and recognition, pattern recognition technology has brought major
changes, significantly improving the early diagnosis and treatment of diseases.\\

\qquad In the pattern recognition technology, 3D image recognition technology has important value. Through the shape, size, color and other information, based on the image analysis
and calculation, three-dimensional image recognition can accurately locate the location and size of the disease, and provide accurate basis for doctors' diagnosis and treatment.
In addition, the technology clearly shows the boundaries of the disease in real time, allowing doctors to assess the extent and spread of the disease and accurately measure the
effectiveness of treatment.\\

\qquad 3D image recognition technology also has powerful dynamic monitoring function. Regular examination can understand the changes of the disease in real time, so that doctors can adjust
the treatment plan in time to achieve effective control of the development of the disease. This not only improves medical efficiency and precision, but also brings personalized medical
services one step closer.\\

\qquad Therefore, in this project, we will deeply explore the application of 3D image recognition in medical diagnosis. It is believed that with the progress of pattern recognition technology,
there will be more breakthroughs in the field of medical health in the future, and more benefits to human health and life.

\section{Schedule and Task Distribution}

We divided the project into the following three stage: \\

\textbf{Stage I} - Preprocess \& Segmentation:

\qquad This stage primarily involves the preprocessing of collected 3D medical images to reduce noise, enhance contrast, and improve image quality and accuracy. Preprocessing typically includes
filtering, denoising, sharpening, among other techniques. Once preprocessing is completed, the image undergoes segmentation. Segmentation refers to extracting target areas of interest from complex
medical images using techniques such as boundary detection and pixel classification. For instance, it can segment organs, tumors or other pathological structures. This step is crucial as it
determines the data used for further analysis and refinement.\\

\textbf{Stage II} - Reconstruction:

\qquad Following accurate image segmentation, the subsequent step is reconstruction which utilizes the segmented information to generate a three-dimensional model. These models enable a deeper understanding
and visualization of body structure while visually observing lesion size, location, and shape. They provide a foundation for precise localization and disease diagnosis.\\

\textbf{Stage III} - Further Application:

\qquad In this final stage, utilizing the 3D models obtained in the previous steps allows for more comprehensive data analysis and interpretation to deliver accurate personalized medical care. For example,
through detailed analysis of three-dimensional reconstructed images doctors can clearly identify specific lesion locations with greater precision while inferring possible development trends; thereby providing
robust support for treatment plan formulation. Moreover these diagnostic results can be analyzed alongside other data such as electronic medical records and laboratory results creating opportunities for big
data application in healthcare.\\

For the project arrangement we based on the following scheme:

\qquad We have six weeks to prepare the project until the project presentation time. When the project report is due to be submitted, we have prepared for Stage I and Stage II for a week.
The schedule and distribution of tasks are as follows:

$$
    \begin{tabular}{|l|l|l|}
        \hline
        Week & Stage                      & Main Contributor \\
        \hline
        1-3  & Preprocess \& Segmentation & Hu Ruilizhen     \\
        \hline
        1-4  & Reconstruction             & Hong Kun         \\
        \hline
        4-6  & Further Application        & Hong Yixi        \\
        \hline
    \end{tabular}
$$

\section{Work Already Done and Further Plan}

\subsection{Work Already Done: 3D Reconstruction from CT-Scan}

1.Differences between CT Images and Traditional Photo Reconstruction:

\qquad In the research process, we first conducted an extensive literature review of the field of three-dimensional reconstruction based on CT images. The general steps of three-dimensional
reconstruction, including preprocessing, segmentation, point cloud generation, and mesh model construction, were examined. We realized that there are significant differences between CT images
and traditional photo reconstruction. Traditional methods such as VisualSFM primarily reconstruct through matrix mapping of appearance features$^{[1]}$, while CT images contain a large amount of
internal information. Therefore, we need different preprocessing and segmentation methods tailored to CT images.\\

2.Lack of Research Combining Segmentation and Reconstruction

\qquad We observed a lack of research currently combining segmentation and reconstruction. We found a study that employed region-growing methods for image segmentation and established a point cloud
information set. In their paper, they mentioned using artificial neural networks to segment images, but due to the overall high costs of establishing datasets, training, and inference, it was
not implemented$^{[2]}$. Subsequently, we explored some new biomedical image segmentation methods, such as U-Net and improved I-Net$^{[3][4]}$. These neural networks efficiently perform segmentation tasks
with a small amount of data, but they are more commonly applied to the analysis of planar images rather than three-dimensional reconstruction.\\

3.Specifics of CT Scanners

\qquad Furthermore, we discovered that certain CT scanners can obtain densely spaced cross-sectional slices and achieve three-dimensional reconstruction through methods like MIP, VRT, MRP, CPR, etc.
Taking VRT as an example, aligning the density values of slices and interpolating can yield three-dimensional voxel data containing coordinate information and density values$^{[5]}$. However, these
reconstructions typically exist in voxel form and are only displayed in specialized software. We believe that converting this voxel data into a generic three-dimensional mesh form will facilitate
further applications and model research. After investigation, we found that data generated by VRT is suitable for processing with MeshLab, software capable of mapping voxel information to mesh models.
We plan to delve deeper into this direction in subsequent research to achieve more widespread applications.\\

4.Conclusion and Outlook

\qquad In conclusion, we have outlined the progress of three-dimensional reconstruction research based on CT images in this mid-term report. Current research mainly focuses on technical exploration in different
steps, but there is a lack of integration between segmentation and reconstruction. Through exploring the specificity of CT scanners and appropriate data processing tools, we hope to more effectively
transform CT images into generic three-dimensional mesh models in future research, thereby advancing this field.\\


\subsection{Further Plan: Preprocess, Segmentation and Application of 3D imaging}

1.Medical Image Segmentation

\qquad The purpose of medical image segmentation is to make changes in anatomical or pathological structures in the image clearer. The segmentation refers to assigning semantics to each voxel/pixel in the medical image$^{[6]}$.
It plays a crucial role in computer-aided diagnosis and intelligent medicine, greatly improving the efficiency and accuracy of diagnosis. It is usually been divided into organ segmentation(distinguishing liver$^{[7]}$, trachea$^{[8]}$,
etc.) and abnormal segmentation(distinguishing liver cancer$^{[9]}$, pulmonary nodules$^{[10]}$,  etc.).\\

2. Medical image detection

\qquad Medical image detection refers to locating a single or multiple regions of interest (ROI) from a large visual range. ROI is a subset of images suitable for expected analysis, manually identified by experts. Nowadays
with the development of AI techniques , ROI detection tools can significantly reduce the number of pixels to be processed, accelerate analysis speed, improve accuracy, and reduce dependence on pathologists.
The detection task is one of the core contents of radiologists' daily film reading, such as target organ registration on abdomen$^{[11]}$. In actual medical screening and diagnosis, only detection is often insufficient, and subsequent
tasks such as classification and segmentation are necessary to complete medical diagnostic reports in clinical practice.\\

3.Classification

\qquad In medical imaging diagnosis, from the diagnosis of benign and malignant risks to the judgment of various signs, classification is usually a high-frequency task carried out by clinical doctors on a daily basis.
For example, taking the diagnosis of lung cancer as an example, doctors usually need to determine the risk of related lesions. In addition, they also need to comprehensively determine the patient's treatment plan
(follow-up, surgery, radiotherapy, chemotherapy, etc.) based on the size of the lesion, the signs of the lesion (burrs, ground glass, etc.), and the location of the lesion (whether it is located next to the lung hilum).
In medical imaging tasks, classification tasks often involve multi label classification, which attempts to label the same entity at different semantic levels.\\

4.Medical image registration

\qquad Medical image registration is a common problem in medical image imaging and display, mainly referring to matching pixels in two medical images with similar structures, textures, etc$^{[12]}$.
Registration is a practical core issue in many medical scenarios, such as how to automatically match multiple examinations for follow-up patients to save clinical doctors' time and facilitate subsequent quantitative measurements
(changes in volume and density).\\

5.3D reconstruction for three-dimensional visualization of lesion areas

\qquad Observing two-dimensional CT images to diagnose a patient's condition is currently one of the main diagnostic surgeries for doctors. However, two-dimensional tomographic images have problems such as aliasing, noise, artifacts,
and mixed effects of sampling points, and cannot display the three-dimensional spatial structure of the lesion area, which poses certain difficulties for doctors to evaluate the condition. Establishing a "realistic" model through
three-dimensional reconstruction technology, segmenting and identifying various tissue structures such as organs, tumors, blood vessels, nerves, and bones in the patient's lesion area, achieving three-dimensional visualization of
the lesion area, facilitating doctors' observation and diagnosis, and conducting digital simulation surgical operations to optimize the surgical plan. The digital simulation comparison before and after surgery can predict the surgical
effect and test the surgical design plan.\\

6.Realizing Accurate Data Measurement through 3D Reconstruction$^{[13]}$

\qquad Due to limited selection of measurement indicator points, single layer evaluation, and unclear images, two-dimensional tomographic images often hinder doctors from accurately measuring data and diagnosing the condition.
The three-dimensional model of the lesion area established through three-dimensional reconstruction technology accurately segments and identifies the contours of various organizational structures, facilitating precise measurement of
the volume and relative position of each anatomical part of the lesion area, guiding doctors to avoid damaging important blood vessels and nerves during surgery. In surgical operation simulation, the use of three-dimensional reconstruction
models can accurately measure the amount of bone removed and the variables of bone block movement, guiding doctors to perform osteotomy more accurately.

\section{References}

 [1] Vacca, G.: OVERVIEW OF OPEN SOURCE SOFTWARE FOR CLOSE RANGE PHOTOGRAMMETRY, Int. Arch. Photogramm. Remote Sens. Spatial Inf. Sci., XLII-4/W14, 239–245, https://doi.org/10.5194/isprs-archives-XLII-4-W14-239-2019\\

[2] Valentin Leonardi, Vincent Vidal, Jean-Luc Mari, and Marc Daniel. 2011. 3D reconstruction from CT-scan volume dataset application to kidney modeling. In Proceedings of the 27th Spring Conference on Computer Graphics (SCCG '11). Association for Computing Machinery, New York, NY, USA, 111-120.

https://doi.org/10.1145/2461217.2461239\\

[3] Ronneberger, O., Fischer, P., Brox, T. (2015). U-Net: Convolutional Networks for Biomedical Image Segmentation. In: Navab, N., Hornegger, J., Wells, W., Frangi, A. (eds) Medical Image Computing and Computer-Assisted Intervention – MICCAI 2015. MICCAI 2015. Lecture Notes in Computer Science(), vol 9351. Springer, Cham. https://doi.org/10.1007/978-3-319-24574-4\_28\\

[4] W. Weng and X. Zhu, "INet: Convolutional Networks for Biomedical Image Segmentation," in IEEE Access, vol. 9, pp. 16591-16603, 2021, doi: 10.1109/ACCESS.2021.3053408.\\

[5] Paul S. Calhoun, Brian S. Kuszyk, David G. Heath, Jennifer C. Carley, and Elliot K. Fishman, “Three-dimensional Volume Rendering of Spiral CT Data: Theory and Method”, RadioGraphics 1999 19:3, 745-764, doi: 10.1148/radiographics.19.3.g99ma14745\\

[6] [医学图像分割综述] Medical Image Segmentation Using Deep Learning: A Survey\_XL\_Dylan的博客-CSDN博客\\

[7] Automated 3D liver segmentation from hepatobiliary phase MRI for enhanced preoperative planning | Scientific Reports (nature.com)\\

[8] ALTIS: A fast and automatic lung and trachea CT‐image segmentation method - Sousa - 2019 - Medical Physics - Wiley Online Library\\

[9] Research on liver cancer segmentation method based on PCNN image processing and SE-ResUnet | Scientific Reports (nature.com)\\

[10] Deep neural network pulmonary nodule segmentation methods for CT images: Literature review and experimental comparisons - ScienceDirect\\

[11] Target organ non-rigid registration on abdominal CT images via deep-learning based detection - ScienceDirect\\

[12]【论文速递11-8】医学图像配准方向优质论文及其代码 - 知乎 (zhihu.com)\\

[13] 数字化三维技术在正畸诊疗中的应用进展\_正颌手术\_三维数字化\_口腔科\_医脉通 (medlive.cn)

\end{document}