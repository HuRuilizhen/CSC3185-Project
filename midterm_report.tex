%----------------------------------------------------------------------------------------
%	PACKAGES AND OTHER DOCUMENT CONFIGURATIONS
%----------------------------------------------------------------------------------------

% Pakcages
\documentclass[12pt,a4paper]{report}
\usepackage[english]{babel}
\usepackage[utf8x]{inputenc}
\usepackage{amsmath}
\usepackage{graphicx}
\usepackage{float}
\usepackage[colorinlistoftodos]{todonotes}
\usepackage{listings}
\usepackage{xcolor}
\usepackage{tikz} \usetikzlibrary{shapes,arrows,positioning, arrows.meta}
\usepackage{fancyhdr}
\usepackage{titlesec}
\usepackage{tocloft}
\usepackage{subcaption}
\usepackage{setspace}
\usepackage[a4paper,top=3cm,bottom=2cm,left=3cm,right=3cm,marginparwidth=1.75cm]{geometry}

% Define custom colors
\definecolor{ghbggray}{HTML}{F6F8FA}
\definecolor{ghblue}{HTML}{0366D6}
\definecolor{ghgreen}{HTML}{22863A}
\definecolor{ghred}{HTML}{B31D28}
\definecolor{ghdarkgray}{HTML}{24292E}
\definecolor{ghorange}{HTML}{D73A49}

% Define code listing settings
\lstset{
    backgroundcolor=\color{ghbggray},
    commentstyle=\color{ghgreen},
    keywordstyle=\color{ghblue},
    numberstyle=\tiny\color{ghdarkgray},
    stringstyle=\color{ghred},
    basicstyle=\ttfamily\footnotesize\color{ghdarkgray},
    breakatwhitespace=false,
    breaklines=true,
    captionpos=b,
    keepspaces=true,
    numbers=left,
    numbersep=5pt,
    showspaces=false,
    showstringspaces=false,
    showtabs=false,
    tabsize=2,
    identifierstyle=\color{ghorange}
}

% Define custom page style
\renewcommand{\thesection}{\arabic{section}}
\renewcommand{\cftsecfont}{\bfseries}
\titleformat{\section}[block]{\normalfont\Large\bfseries}{\thesection}{1em}{}
\setlength{\parindent}{0pt}

% Define line spacing
\setstretch{1}

%----------------------------------------------------------------------------------------
%	COVER
%----------------------------------------------------------------------------------------

\begin{document}

\begin{titlepage}

    \newcommand{\HRule}{\rule{\linewidth}{0.5mm}}

    \center
    \vspace*{1.5cm}

    %----------------------------------------------------------------------------------------
    %	HEADING SECTIONS
    %----------------------------------------------------------------------------------------

    \includegraphics[scale=.2]{src/cuhk.png}\\[1cm]
    \textsc{\large The Chinese University of Hong Kong, Shenzhen}\\[1.5cm]

    %course code
    \textsc{\Large CSC3185}\\[0.5cm]

    %course name
    \textsc{\large Introduction to Multimedia Systems}\\[0.5cm]

    %----------------------------------------------------------------------------------------
    %	TITLE SECTION
    %----------------------------------------------------------------------------------------

    \HRule \\[0.4cm]
    { \bfseries Application of 3D Image Recognition in Medical Diagnosis}
    \HRule \\[1.5cm]

    %----------------------------------------------------------------------------------------
    %	AUTHOR SECTION
    %----------------------------------------------------------------------------------------

    \begin{minipage}{0.6\textwidth}
        \begin{tabular}{l l l}
            \emph{Group ID:}     & 4                        \\
            \emph{Group Member:} & Hong Kun     & 120040069 \\
                                 & Hu Ruilizhen & 122090168 \\
                                 & Hong Yixi    & 119010101 \\
        \end{tabular}

    \end{minipage}\\[2cm]

    %----------------------------------------------------------------------------------------
    %	DATE SECTION
    %----------------------------------------------------------------------------------------

    % Date
    {\large \today}\\[2cm]
    \vfill
\end{titlepage}

%----------------------------------------------------------------------------------------
%	CONTENTS
%----------------------------------------------------------------------------------------

% \tableofcontents
% \newpage

%----------------------------------------------------------------------------------------
%	HEADER AND FOOTER
%----------------------------------------------------------------------------------------

\fancypagestyle{mypagestyle}{
    \fancyhf{}
    \fancyhead[L]{\small CSC3185 Midterm Report}
    \fancyhead[R]{\small Group 4}
    \fancyfoot[C]{\thepage} % Add this line to include page numbers at the center of the footer
    \renewcommand{\headrulewidth}{1pt}
}

% Apply custom page style
\pagestyle{mypagestyle}

%----------------------------------------------------------------------------------------
%	MAIN BODY
%----------------------------------------------------------------------------------------

\section{Topic and Understanding}

\qquad In recent years, the application of pattern recognition technology has penetrated into various fields, opening up new possibilities for the development of artificial intelligence.
One of the most influential and dynamic application areas is in the field of health care. Especially in disease detection and recognition, pattern recognition technology has brought major
changes, significantly improving the early diagnosis and treatment of diseases.\\

\qquad In the pattern recognition technology, 3D image recognition technology has important value. Through the shape, size, color and other information, based on the image analysis
and calculation, three-dimensional image recognition can accurately locate the location and size of the disease, and provide accurate basis for doctors' diagnosis and treatment.
In addition, the technology clearly shows the boundaries of the disease in real time, allowing doctors to assess the extent and spread of the disease and accurately measure the
effectiveness of treatment.\\

\qquad 3D image recognition technology also has powerful dynamic monitoring function. Regular examination can understand the changes of the disease in real time, so that doctors can adjust
the treatment plan in time to achieve effective control of the development of the disease. This not only improves medical efficiency and precision, but also brings personalized medical
services one step closer.\\

\qquad Therefore, in this project, we will deeply explore the application of 3D image recognition in medical diagnosis. It is believed that with the progress of pattern recognition technology,
there will be more breakthroughs in the field of medical health in the future, and more benefits to human health and life.

\section{Schedule and Task Distribution}

We divided the project into the following three stage: \\

\textbf{Stage I} - Preprocess \& Segmentation:

\qquad This stage primarily involves the preprocessing of collected 3D medical images to reduce noise, enhance contrast, and improve image quality and accuracy. Preprocessing typically includes
filtering, denoising, sharpening, among other techniques. Once preprocessing is completed, the image undergoes segmentation. Segmentation refers to extracting target areas of interest from complex
medical images using techniques such as boundary detection and pixel classification. For instance, it can segment organs, tumors or other pathological structures. This step is crucial as it
determines the data used for further analysis and refinement.\\

\textbf{Stage II} - Reconstruction:

\qquad Following accurate image segmentation, the subsequent step is reconstruction which utilizes the segmented information to generate a three-dimensional model. These models enable a deeper understanding
and visualization of body structure while visually observing lesion size, location, and shape. They provide a foundation for precise localization and disease diagnosis.\\

\textbf{Stage III} - Further Application:

\qquad In this final stage, utilizing the 3D models obtained in the previous steps allows for more comprehensive data analysis and interpretation to deliver accurate personalized medical care. For example,
through detailed analysis of three-dimensional reconstructed images doctors can clearly identify specific lesion locations with greater precision while inferring possible development trends; thereby providing
robust support for treatment plan formulation. Moreover these diagnostic results can be analyzed alongside other data such as electronic medical records and laboratory results creating opportunities for big
data application in healthcare.\\

For the project arrangement we based on the following scheme:

\qquad We have six weeks to prepare the project until the project presentation time. Each phase of the project will take two weeks. When the project report is due to be submitted, we have prepared for Stage I for a week.
The schedule and distribution of tasks are as follows:

$$
    \begin{tabular}{|l|l|l|}
        \hline
        Week & Stage                      & Main Contributor \\
        \hline
        1-2  & Preprocess \& Segmentation & Hong Kun         \\
        \hline
        3-4  & Reconstruction             & Hu Ruilizhen     \\
        \hline
        5-6  & Further Application        & Hong Yixi        \\
        \hline
    \end{tabular}
$$

\section{Work Already Done and Further Plan}

\end{document}